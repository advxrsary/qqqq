\documentclass{article}

\usepackage{amsmath}
\usepackage{amssymb}

\begin{document}

\title{COMP2300/COMP6300 Applied Cryptography \\ Assignment 2}
\date{Deadline: Friday (Week 12), 26 May 2023 (17:00 pm)}
\maketitle

\textbf{Total marks:} 30

\textbf{Weighting:} 15\%

\textbf{Note:} Submit the assignment via iLearn (Include Student Name and ID in assignment).

\section*{Objectives}
This assignment has been designed to test your knowledge of public-key cryptography and application of cryptography in cryptocurrencies.

\section*{Notes}
\begin{itemize}
\item Assumptions (if any) must be stated clearly in your answers.
\item Please upload two files only: A text file (Word or PDF) answering the descriptive/numerical parts of the questions, and a .java or .py file containing the programming parts (Question 3, parts (a) and (b)). Combine the code for Question 3, parts (a) and (b) into one program.
\item Where a proof is required, showing how the statement is true by giving one particular example does not constitute a proof. No marks will be granted in such a case.
\end{itemize}

\section*{Submission}
\begin{itemize}
\item Online submission via iLearn. Assignments will be marked and returned online. There are no hardcopy submissions for written assignments.
\end{itemize}


\section{Question 1 (10 marks)}
In this question, we are going to explore the Pohlig-Hellman algorithm to solve the discrete logarithm problem (DLP) in a group. Roughly, the algorithm exploits the fact that the DLP problem in a group G is only as hard as the DLP problem in the largest subgroup of G [Sma16, §3.2]. For this exercise, we shall assume that G is a finite multiplicative group. The following definitions and results are from [Gal21].


\textit{Subgroups.} A subset H of a group G is called a subgroup of G if it is itself a group. Luckily, we do not need to check all the group axioms to verify if a subset H of G is a subgroup. Instead, we can use the following result:


\textit{Theorem 1.} Let G be a group and H a nonempty subset of G. If $ab^{-1}$ is in H whenever a and b are in H, then H is a subgroup of G.


\textit{Cyclic Groups.} A group G is called cyclic if there is an element $g \in G$ such that $G = \{g^{i} : i \in \mathbb{Z}\}$. This g is called the generator of G. We write $\langle g \rangle$. We can define this for any element $a \in G$, i.e., $\langle a \rangle = \{a^{i} : i \in \mathbb{Z}\}$. Notice that even though the powers of a go through all integers (which is of infinite cardinality), the number of elements in the group, i.e., its order, is finite. In particular, for any positive integer i we define: 
\begin{equation*}
    a^{-i} = (a^{i})^{-1}
\end{equation*}

(a) Show that for any $a \in G$, $\langle a \rangle$ is a subgroup of G. (2 marks)

Let us now assume that G has order n, where n = pq is a product of two primes. Let g be a generator of G. 

(b) Show that the order of the subgroup $\langle g^{p} \rangle$ is q. (2 marks)

Similarly, the order of $\langle g^{q} \rangle$ is p.

\textit{Pohlig-Hellman Attack.} Let us look at a concrete example of such a group. Let $N$ be a composite number. Let $\mathbb{Z}_{N}^{*}$ be the group of all positive integers relatively prime to $N$. Then this set together with multiplication modulo $N$ forms a group. Let $n = pq$ be the order of this group. Note that $n$ is different from $N$. Suppose we are given: 
\begin{equation*}
    h \equiv g^x \pmod{N}
\end{equation*}
where $g$ is the generator of
$\mathbb{Z}_{N}^{*}$. This is a DLP instance. We can raise the above to the powers $p$ and $q$ to obtain 
\begin{equation}
\begin{aligned}
h^p &\equiv (g^p)^x \pmod{N}, \\
h^q &\equiv (g^q)^x \pmod{N}
\end{aligned}
\end{equation}

Note that $g^p$ has order $q$ and $g^q$ has order $p$ as you proved above. Eq. 1 shows two DLP problems in these two subgroups of $\mathbb{Z}_{N}^{*}$. Let us assume that we can solve DLP in these subgroups. This means we know: 
\begin{align*}
x \equiv a_1 \pmod{p}, \\
x \equiv a_2 \pmod{q}
\end{align*}


Then, via the Chinese remainder theorem, we can obtain $x$ modulo $n$, i.e., we can solve the original DLP instance. Why would DLP be easier in subgroups? Because they have smaller number of elements, and it may be feasible to solve DLP in these subgroups. This is exactly the point exploited in the Pohlig-Hellman algorithm.

\begin{itemize}
    \item[(c)] Let $N = 18$. The order of $\mathbb{Z}_{18}^{*}$ is $6$. Enumerate all the elements of $\mathbb{Z}_{18}^{*}$. (2 marks)
    \item[(d)] Let the DLP instance in $\mathbb{Z}_{18}^{*}$ be 
    \begin{align*}
    7 \equiv 5^x \pmod{18}
    \end{align*}
    Since $n = 6 = 2\times3$, let $p = 2$, and $q = 3$. What are the elements generated by $5^2 \pmod{18}$ and $5^3 \pmod{18}$? (2 marks)
    \item[(e)] By raising 
    \begin{align*}
    7 \equiv 5^x \pmod{18}
    \end{align*}
    to the powers $2$ and $3$, we get
    \begin{align*}
    13 \equiv 7^x \pmod{18}\text{ and } 1\equiv 17^x \pmod{18}.
    \end{align*}
    
    respectively. Solving DLP in the first subgroup gives us $x \equiv 2 \pmod{3}$. What is $x$ in the second subgroup? Use the Chinese Remainder Theorem (using PARI) to find $x$ in the original DLP instance. (2 marks)
\end{itemize}
\section{Question 2 (10 marks)}
For reasons, that will become clear, the version of RSA encryption covered in the lecture is called “naive RSA” in [Sma16]. In practice, it is recommended that RSA be used with some sort of message padding. A standard way is using optimal asymmetric encryption padding (OAEP), as indicated in the lecture. 

In this question, we will cover a simpler version, which at least makes the scheme semantically secure under chosen plaintext attacks. The CEO of the organization XYZ decides to hold a vote to decide whether employees should be allowed to work from home (WFH) either one, two or three days a week. All 4 employees of XYZ (excluding the CEO), need to vote either “WFH one” , “WFH two” or “WFH three”. 

To ensure privacy, the CEO asks employees to send their votes by emailing them to the CEO. Furthermore, the CEO tells them to encrypt their votes using her RSA public key $e$. The messages are encoded as: 
\begin{center}
WFH one $7\rightarrow 100$,

WFH two $7\rightarrow 200$,

WFH three $7\rightarrow 300$.
\end{center}
The CEO receives the sequence of ciphertexts: 
\begin{center}
$c1, c2, c3, c4$
\end{center}
where $c_i$ is the ciphertext from employee $i$. For completeness, assume the employees are in the order: 
\begin{center}
Alice, Bob, Charlie, Dave.
\end{center} So, e.g., Bob’s ciphertext is $c_2$.

\begin{itemize}
    \item[(a)] How many possible sequences of votes are there and why? (2 marks)
    \item[(b)] Assume you, the eavesdropper, get the sequence: 
    \begin{center}
208149, 249575, 272202, 249575
\end{center} After observing this, how many possible sequences of votes are there and why? (2 marks)
    \item[(c)] Suppose the CEO’s RSA public key is $(N, e) = (311119, 11)$. What is Charlie’s vote? How did you come to this conclusion? Your answer should not involve any decryption. (2 marks)
\end{itemize}

The issue raised above is that naive RSA is deterministic. We would want different ciphertexts, even if the same plaintext is encrypted twice. To do this, instead of letting $m$ be from the set ${0, 1, . . . , N − 1} = Z_N$, we will restrict it to a smaller set, and then use the remaining bits for a random number $r$. We will then encrypt $r||m$ instead of $m$. Let $r \in {0,1}^l$. Let $|N|$ be the number of bits in the binary representation of $N$.

Then, encryption of the message $m$ in this “padded” version of RSA is: 
\begin{enumerate}
    \item Choose a random $r \in {0, 1}^l$.
    \item Encrypt as $(r||m)^e \pmod{N}$.
\end{enumerate}
Decryption after receiving $c$ is:
\begin{enumerate}
    \item Decrypt as $(c)^d \pmod{N}$.
    \item Retrieve the last $|N| − l$ bits as the message $m$.
\end{enumerate}
In practice, we will first convert both $r$ and $m$ into their integer representations, and then encrypt $r||m$.

\begin{itemize}
    \item[(d)] Suppose $l = 3$, so $r$ can be from the set ${0, 1, 2, . . . , 7}$. Assume you, the eavesdropper, gets the sequence: 
    \begin{center}
    289509, 3961, 104376, 175767
    \end{center} 
    encrypted via the padded RSA scheme using $(N, e) = (311119, 11)$. Without decrypting, find the sequence of votes. (2 marks)
    \item[(e)] Suppose now that $l = 100$, so $r$ can be from the set ${0,1}^{100}$. What is the time complexity of your attack from part (d)? Is the attack feasible? If you were an employee, would you recommend this encryption scheme to the CEO? (2 marks)
\end{itemize}

A real-world account of a similar form of data release using (deterministically) encrypted IDs can be found here [CRT17].


\section{Question 3 (10 marks)}
Recall hash puzzles from Lecture 10. In this question, we will see how they work in practice. This question requires programming in a language that has an implementation of SHA-256 hash function. You should use Python’s hashlib for this purpose. 

Create an integer $s$ containing all the digits in your student ID. For example, if my student ID is 12345678, then $s = 12345678$. Let ``str'' denote the string function, i.e., given any integer $s$, the function str($s$) casts it into a string. For example, $s = 12345678$ becomes str($s$) =$``12345678''$. Set the target $t$ to:

\begin{equation*}
t = \frac{2^{256}}{2^8}
\end{equation*}

Let $H$ be the SHA-256 hash function. Let $r$ be a counter starting from 1. Finally, let \verb!||! denote string concatenation.

\begin{itemize}
    \item[(a)] Implement a program that tries successive values of $r$, i.e., $r = 1,2,3,...$, computes $H($\text{str}(r)\verb!||!\text{str}(s)$)$, compares it with $t$ and halts whenever $H($\text{str}(r)\verb!||!\text{str}(s)$) < t$, with the output $r$. You need to provide the program and the output $r$. (5 marks)
    \item[(b)] Let us call the program from part (a) as: PuzzleFinder($s$, $t$). Write a program that calls PuzzleFinder with successive inputs ($s + i$, $t$), for $i = 0$ to $999$, and records the output for each of these 1,000 runs. What is the average number of attempts in these 1,000 runs before you found the target? You need to provide your program. (3 marks)
    \item[(c)] If you want someone to take $2^{60}$ attempts on average before they solve the puzzle, what should you set the target to and why? (2 marks)
\end{itemize}



\end{document}
